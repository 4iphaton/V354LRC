\section{Auswertung}
\label{sec:Auswertung}

\begin{figure}
  \centering
  \includegraphics{plot1.pdf}
  \caption{Plot1.}
  \label{fig:plot1}
\end{figure}

\begin{figure}
  \centering
  \includegraphics{plot2.pdf}
  \caption{Plot2.}
  \label{fig:plot2}
\end{figure}


\subsection{5a}
\begin{equation}
   a = \num{-2.65(15)}\, \cdot\, 10^{4}
\end{equation}

\begin{equation}
   T_{\text{ex}} = \SI{3.78(21)e5}{\per\second}
\end{equation}

\begin{equation}
R_{\text{eff}} = \SI{535(30)}{\ohm}
\end{equation}


\subsection{5b}

\begin{equation}
  R_{\text{ap,theo}} = \SI{4390(9)}{\ohm}
  \end{equation}
\begin{equation}
  R_{\text{ap,exp}} = \SI{3400}{\ohm}
\end{equation}


\subsection{5c}

\begin{equation}
  q_{\text{exp}} = 3.9
\end{equation}
\begin{equation}
  q_{\text{theo}} = \num{3.923(9)}
\end{equation}



\begin{equation}
  \nu_{-} = 28900
\end{equation}
\begin{equation}
  \nu_{+} = 38300
\end{equation}

daraus folgt 9400
\begin{equation}
  (\nu_{+}-\nu_{-})_{\text{exp}} = \SI{9400}{\Hertz}
\end{equation}
\begin{equation}
  (\nu_{+}-\nu_{-})_{\text{theo}} = \SI{8808(27)}{\Hertz}
\end{equation}
