\section{Auswertung}
\label{sec:Auswertung}

\subsection{verwendete Bauteile}

Die verwendeten Werte der Bauteile sind:
\begin{equation*}
  L = \SI{10.11(3)e-3}{\henry}
\end{equation*}
\begin{equation*}
  C = \SI{2.098(6)e-9}{\farad}
\end{equation*}
\begin{equation*}
  R_{\text{1}} = \SI{48.1(1)}{\ohm}
\end{equation*}
\begin{equation*}
  R_{\text{2}} = \SI{509.5(5)}{\ohm}.
\end{equation*}

\subsection{Messung A, gedämpfte Schwingung}

\begin{figure}
  \centering
  \includegraphics{plot1.pdf}
  \caption{Einhüllende.}
  \label{fig:plot1}
\end{figure}
 Die positiven und negierten negativen Amplituden der Spannung $U$ werden gemeinsam
gegen die Zeit $t$ in s aufgetragen (Abb.\ref{fig:plot1}) und dann ein Regressionsgraph
unter Zuhilfenahme von Python mit
\begin{equation*}
  exp(a*x +b)
\end{equation*}
erstellt dessen
\begin{equation*}
   a = \num{-2.65(15)}\, \cdot\, 10^{4}
\end{equation*}
dem Exponenten $2\pi\nu$ aus (\ref{eqn:einhuellend}) entspricht.
Hiermit lassen sich nach (\ref{tex}) und (\ref{reff}) mit dem verwendeten Widerstand $R_{\text{2}}$
\begin{equation*}
   T_{\text{ex}} = \SI{3.78(21)e5}{\per\second}
\end{equation*}
und
\begin{equation*}
R_{\text{eff,exp}} = \SI{535(30)}{\ohm}
\end{equation*}
berechnen.

Im Vergleich mit dem verwendenten Widerstand $R_{\text{2}}$ ergibt sich eine Abweichung
von $\SI{5.01}{\percent}$ die durch den Eigenwiderstand der Schaltung wie z.B. den Kondensator
erklären lässt.

\subsection{Messung B, Dämpfungswiderstand}
Der theoretische Dämpfungswiderstand wird mit (\ref{eqn:R11}) zu
\begin{equation}
  R_{\text{ap,theo}} = \SI{4390(9)}{\ohm}
  \end{equation}
  berechen, während der gemessene experimentelle Wert
\begin{equation}
  R_{\text{ap,exp}} = \SI{3400}{\ohm}
\end{equation}
entspricht.
Die prozentuale Abweichung ergibt sich als $\SI{22.55}{\percent}$ und ist durch
die generelle Unsicherheit des Ablesens vom Oszilloskop, durch das Verstellen
des variablen Widerstandes per Hand und den Eigenwiderstand der Schaltung  zu erklären.

\subsection{Messung C, Frequenzabhängigkeit der Kondensatorspannung}

Die Erregerspannung $U_{Erreger}$ beträgt 1V und wenn man $\frac{U_{C}}{U_{Erreger}}$
ist diese halblogarithmisch möglich, da diese Größe einheitenlos ist (siehe Abb.\ref{fig:plot2}).

\begin{figure}
  \centering
  \includegraphics{plot2.pdf}
  \caption{halblogarithmisch aufgetragen.}
  \label{fig:plot2}
\end{figure}

\begin{figure}
  \centering
  \includegraphics{plot3.pdf}
  \caption{Plot3.}
  \label{fig:plot3}
\end{figure}

\begin{equation}
  q_{\text{exp}} = 3.9
\end{equation}
\begin{equation}
  q_{\text{theo}} = \num{3.923(9)}
\end{equation}

\begin{equation}
  \nu_{-,exp} = 28900
\end{equation}
\begin{equation}
  \nu_{+,exp} = 38300
\end{equation}

\begin{equation}
  (\nu_{+}-\nu_{-})_{\text{exp}} = \SI{9400}{\hertz}
\end{equation}
\begin{equation}
  (\nu_{+}-\nu_{-})_{\text{theo}} = \SI{8808(27)}{\hertz}
\end{equation}

\subsection{5d}

\begin{figure}
  \centering
  \includegraphics{plot4.pdf}
  \caption{Plot4.}
  \label{fig:plot4}
\end{figure}


Theorie:

\begin{equation}
  w_{\text{1,theo}} = \SI{30.43(6)e3}{\hertz}
\end{equation}

\begin{equation}
  w_{\text{2,theo}} = \SI{39.24(8)e3}{\hertz}
\end{equation}

\begin{equation}
  w_{\text{res,theo}} = \SI{8.808(27)e3}{\hertz}
\end{equation}

Experimentell:

\begin{equation}
  w_{\text{1,exp}} = \SI{25e3}{\hertz}
\end{equation}

\begin{equation}
  w_{\text{2,exp}} = \SI{34.55e3}{\hertz}
\end{equation}

\begin{equation}
  w_{\text{res,exp}} = \SI{9.549e3}{\hertz}
\end{equation}
