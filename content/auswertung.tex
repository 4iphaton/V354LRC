\section{Auswertung}
\label{sec:Auswertung}

\subsection{verwendete Bauteile}

Die verwendeten Werte der Bauteile sind:
\begin{equation*}
  L = \SI{10.11(3)e-3}{\henry}
\end{equation*}
\begin{equation*}
  C = \SI{2.098(6)e-9}{\farad}
\end{equation*}
\begin{equation*}
  R_{\text{1}} = \SI{48.1(1)}{\ohm}
\end{equation*}
\begin{equation*}
  R_{\text{2}} = \SI{509.5(5)}{\ohm}.
\end{equation*}
\begin{equation*}
  R_{\text{G}} = \SI{50}{\ohm}.
\end{equation*}

\subsection{Messung A, gedämpfte Schwingung}

\begin{table}[h]
  \centering
  \label{tab:posA}
  \begin{tabular}{ c c }
    \toprule
    {$\text{t} \,\, \text{in} \,\,\si{\second}$}
   &{$\text{U} \,\, \text{in} \,\si{\volt}$} \\

    \midrule
    0       & 12.5 \\
    0.00003 &  5   \\
    0.00006 &  2.25\\
    0.00009 &  1   \\
    \bottomrule
  \end{tabular}
  \caption{positive Amplituden.}
\end{table}

\begin{table}[h]
  \centering
  \label{tab:negA}
  \begin{tabular}{ c c }
    \toprule
    {$\text{t} \,\, \text{in} \,\,\si{\second}$}
   &{$\text{U} \,\, \text{in} \,\si{\volt}$} \\

    \midrule
    0.000015 & -9   \\
    0.000045 & -4.25\\
    0.000075 & -2   \\
    0.000105 & -1   \\
    \bottomrule
  \end{tabular}
  \caption{negative Amplituden.}
\end{table}

\begin{figure}
  \centering
  \includegraphics{plot1.pdf}
  \caption{Einhüllende.}
  \label{fig:plot1}
\end{figure}
 Die positiven und negierten negativen Amplituden der Spannung $U$ werden gemeinsam
gegen die Zeit $t$ in s aufgetragen (Abb.\ref{fig:plot1}) und dann ein Regressionsgraph
unter Zuhilfenahme von Python mit
\begin{equation*}
  \text{exp}(ax+b)
\end{equation*}
erstellt, dessen
\begin{equation*}
   a = \num{-2.65(15)}\, \cdot\, 10^{4}
\end{equation*}
dem Exponenten $2\pi\mu$ aus (\ref{eqn:einhuellend}) entspricht.
Hiermit lassen sich nach (\ref{tex}) und (\ref{reff}) mit dem verwendeten Widerstand $R_{\text{2}}$
\begin{equation*}
   T_{\text{ex}} = \SI{3.78(21)e5}{\per\second}
\end{equation*}
und
\begin{equation*}
R_{\text{eff,exp}} = \SI{535(30)}{\ohm}
\end{equation*}
berechnen.



\subsection{Messung B, Dämpfungswiderstand}
Der theoretische Dämpfungswiderstand wird mit (\ref{eqn:R11}) zu
\begin{equation*}
  R_{\text{ap,theo}} = \SI{4390(9)}{\ohm}
\end{equation*}
  berechen, während der gemessene experimentelle Wert
\begin{equation*}
  R_{\text{ap,exp}} = \SI{3400}{\ohm}
\end{equation*}
entspricht.

\subsection{Messung C, Frequenzabhängigkeit der Kondensatorspannung}

\begin{table}[h]
  \centering
  \label{tab:Uc}
  \begin{tabular}{ c c }
    \toprule
    {$U_{\text{C}} \,\, \text{in} \,\si{\volt}$}
   &{$\nu \,\, \text{in} \,\si{\hertz}$} \\

    \midrule
    1.050  & 10000 \\
    1.175  & 15000 \\
    1.420  & 20000 \\
    1.950  & 25000 \\
    3.000  & 30000 \\
    3.600  & 32000 \\
    3.750  & 33000 \\
    3.900  & 34000 \\
    3.800  & 35000 \\
    3.500  & 36000 \\
    2.850  & 38000 \\
    2.200  & 40000 \\
    1.250  & 45000 \\
    0.730  & 50000 \\
    0.600  & 55000 \\

    \bottomrule
  \end{tabular}
  \caption{Kondensatorspannung mit Tastkopf auf x10 gestellt.}
\end{table}

Es wird ein Gesamtwiderstand von $R_\text{2} + R_\text{G} $verwendet.
Die Erregerspannung $U_{\text{Erreger}}$ beträgt 1V und wenn man $\frac{U_{C}}{U_{\text{Erreger}}}$
verwendet ist diese halblogarithmisch auftragbar, da diese Größe einheitenlos ist (siehe Abb.\ref{fig:plot2}).

\begin{figure}
  \centering
  \includegraphics{plot2.pdf}
  \caption{halblogarithmisch aufgetragen.}
  \label{fig:plot2}
\end{figure}

Mit(\ref{eqn:qexp}) wird anschließend die Güte experimentell bestimmt und es ergibt sich:
\begin{equation*}
  q_{\text{exp}} = 3.9.
\end{equation*}
Worauf sich der theoretische Wert durch (\ref{eqn:qtheo}) berechnen lässt.
\begin{equation*}
  q_{\text{theo}} = \num{3.923(9)}.
\end{equation*}

\begin{figure}
  \centering
  \includegraphics{plot3.pdf}
  \caption{Bestimmung der Breite.}
  \label{fig:plot3}
\end{figure}

In Abb.\ref{fig:plot3} wird der Plot vergrößert und eine Linie auf der Höhe
$\frac{U_{C,max}}{\sqrt{2}}$ eingefügt. Dann werden die Messwerte linear
verbunden um $\nu_{-,exp}\, \text{und}\,\nu_{+,exp}$ abzulesen. Damit dies
genauer ist werden jeweils noch 2 weitere Linien hinzugefügt um besser an der y-Achse
ablesen zu können.
Die abzulesenden Werte und somit auch die Positionen der Linien sind
\begin{equation*}
  \nu_{-,exp} = 28900
\end{equation*}
\begin{equation*}
  \nu_{+,exp} = 38300
\end{equation*}
woraus sich die Breite
\begin{equation*}
  (\nu_{+}-\nu_{-})_{\text{exp}} = \SI{9400}{\hertz}
\end{equation*}
ergibt. Wohingegen die mit (\ref{eqn:breite}) berechnete Breite
\begin{equation*}
  (\nu_{+}-\nu_{-})_{\text{theo}} = \SI{8808(27)}{\hertz}
\end{equation*}
ist.

\subsection{Messung D, Frequenzabhängigkeit der Phase}

\begin{table}[h]
  \centering
  \label{tab:Uc2}
  \begin{tabular}{ c c }
    \toprule
    {$\Phi \,\, \text{in} \,\si{\mu\second}$}
   &{$\nu \,\, \text{in} \,\si{\hertz}$} \\

    \midrule
    2.00    & 10000 \\
    2.00    & 15000 \\
    2.00    & 20000 \\
    2.50    & 25000 \\
    4.00    & 30000 \\
    5.00    & 32000 \\
    6.00    & 33000 \\
    7.00    & 34000 \\
    7.75    & 35000 \\
    8.50    & 36000 \\
    9.50    & 38000 \\
    10.00   & 40000 \\
    10.00   & 45000 \\
    9.50    & 50000 \\
    9.00    & 55000 \\


    \bottomrule
  \end{tabular}
  \caption{Messwerte Phase.}
\end{table}

Bei dieser Messung wird die Phase $\Phi$ im Bogenmaß gegen die Frequenzen $\nu$ in kHz aufgetragen
und jeweils erneut Geraden für in den Plot (siehe Abb.\ref{fig:plot4}) eingebracht um die
experimentellen $\nu$ abzulesen.
\begin{figure}[H]
  \centering
  \includegraphics{plot4.pdf}
  \caption{Frequenzabhängigkeit der Phase.}
  \label{fig:plot4}
\end{figure}
Abzulesen ist dann
\begin{equation*}
  \nu_{\text{1,exp}} = \SI{25e3}{\hertz}
\end{equation*}
für $\phi = \frac{\pi}{4}$,
\begin{equation*}
  \nu_{\text{2,exp}} = \SI{34.55e3}{\hertz}
\end{equation*}
für $\phi = \frac{3\pi}{4}$ und
\begin{equation*}
  \nu_{\text{res,exp}} = \SI{32e3}{\hertz}
\end{equation*}
für $\phi = \frac{\pi}{2}$.

Mit (\ref{o12}) und (\ref{eqn:ores}) ergeben sich unter Beachtung von (\ref{c1})
die folgenden theoretischen Werte der unterschiedlichen $\nu$:
\begin{equation*}
  \nu_{\text{1,theo}} = \SI{30.43(6)e3}{\hertz}
\end{equation*}
\begin{equation*}
  \nu_{\text{2,theo}} = \SI{39.24(8)e3}{\hertz}
\end{equation*}
\begin{equation*}
  \nu_{\text{res,theo}} = \SI{34.56(7)e3}{\hertz}.
\end{equation*}




%\begin{equation*}
%  w_{\text{res,theo}} = \SI{8.808(27)e3}{\hertz}
%\end{equation*}

%vgl omega1 - omega2 und finde breite omega_res


%\begin{equation*}
%  w_{\text{res,exp}} = \SI{9.549e3}{\hertz}
%\end{equation*}
