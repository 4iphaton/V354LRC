\section{Diskussion}
\label{sec:Diskussion}

\begin{table}[h]
  \centering
  \label{tab:abw}
  \begin{tabular}{ c c }
    \toprule
   $\text{physikalische Größe}$
   &{$\text{prozentuale Abweichung}\,\, \text{in} \, \si{\percent} $} \\

    \midrule
    $R_{\text{eff}}$ & 5.01 \\
    $R_{\text{ap}}$& 22.55  \\
    $q$ & 0.59 \\
    $\nu_{+}-\nu_{-}$ & 6.72 \\
    $\nu_{\text{1}}$ & 17.84\\
    $\nu_{\text{2}}$ & 11.95 \\
    $\nu_{\text{res}}$ & 7.41\\
    \bottomrule
  \end{tabular}
  \caption{Abweichung von Theoriewert zu experimentellem Wert.}
\end{table}

Bei Messung A ergibt sich im Vergleich mit dem verwendenten Widerstand $R_{\text{2}}$
eine Abweichung von $\SI{5.01}{\percent}$ die durch den Eigenwiderstand des Stromkreises zu erklären ist.
Außerdem wurde die Messung mit dem größeren der beiden Widerstände durchgeführt, was sie
eventuell auch weiter abweichen lassen hat.
Die prozentuale Abweichung von $R_{\text{ap}}$ aus Messung B ergibt sich als
$\SI{22.55}{\percent}$ und ist durch die generelle Unsicherheit des Ablesens des Überschwingens, welches auf dem
Oszilloskop abzulesen ist, durch das ungenaue Verstellen des variablen Widerstandes per Hand
und den Eigenwiderstand des Stromkreises zu erklären.
Sämtliche anderen Abweichungen sind akzeptabel, wobei Abweichungen bei den Frequenzen
$\nu$ und der Breite $\nu_{+}-\nu_{-}$ durch Ungenauigkeiten beim Auslesen der
experimentellen Werte aus den verschiedenen Plots zu erklären sind.
